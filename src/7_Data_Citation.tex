\section{Data Management and Citation}

\subsection{Why Should We Cite Data?}
\subsubsection{Preventing Scientific Misconduct}
\begin{itemize}
    \item Data citation helps prevent issues like data fabrication, falsification, and selective reporting.
    \item Studies indicate that 2\% of scientists admit to falsifying data, while 34\% acknowledge questionable research practices.
\end{itemize}

\subsubsection{Giving Credit and Recognition}
\begin{itemize}
    \item Citations are the "currency" of science, incentivizing data sharing by providing recognition for contributors.
\end{itemize}

\subsubsection{Establishing a Solid Basis}
\begin{itemize}
    \item Citing data builds on a foundation of prior work, improving efficiency and enabling scientific discourse.
\end{itemize}

\subsubsection{Facilitating Reproducibility and Reuse}
\begin{itemize}
    \item Proper citation ensures that others can reproduce, verify, and build upon existing research.
\end{itemize}

\subsection{Data Citation Principles}
\begin{itemize}
    \item \textbf{Importance:} Treat data as a citable product of research.
    \item \textbf{Credit and Attribution:} Acknowledge contributors through proper citation.
    \item \textbf{Evidence:} Link claims to specific datasets.
    \item \textbf{Unique Identification:} Use persistent and globally unique identifiers.
    \item \textbf{Access and Persistence:} Ensure access to data and associated metadata over time.
    \item \textbf{Specificity and Verifiability:} Facilitate identification and validation of cited data.
    \item \textbf{Interoperability and Flexibility:} Adaptable citation methods across diverse practices while maintaining consistency.
\end{itemize}

\subsection{Data Management}
\subsubsection{What is Data Management?}
\begin{itemize}
    \item Planning and organizing data throughout its lifecycle, including collection, storage, sharing, and preservation.
\end{itemize}

\subsubsection{Data Management Plans (DMPs)}
\begin{itemize}
    \item Define how data will be created, documented, accessed, and preserved.
    \item Address issues such as ethics, intellectual property, sharing strategies, and long-term storage.
    \item Often required in research and industry settings.
\end{itemize}

\subsubsection{Creating a DMP}
\begin{itemize}
    \item DMP templates guide researchers through questions on data collection, storage, sharing, and preservation.
    \item Automated tools and machine-actionable DMPs (maDMPs) streamline the process.
\end{itemize}

\subsection{Best Practices for Data Citation}
\begin{itemize}
    \item Use persistent identifiers like DOIs or PURLs.
    \item Provide detailed metadata, including dataset descriptions, standards, and sharing guidelines.
    \item Enable reproducibility by specifying the version, granularity, and provenance of the data.
\end{itemize}

\subsection{Dynamic Data Citation}
\subsubsection{Challenges}
\begin{itemize}
    \item Data is often dynamic, with frequent updates or corrections.
    \item Precise identification of subsets and versions is crucial.
\end{itemize}

\subsubsection{Solutions}
\begin{itemize}
    \item Use timestamped queries and versioned datasets.
    \item Assign persistent identifiers to queries, enabling reproducibility and traceability.
    \item Provide landing pages with metadata, access options, and citation recommendations.
\end{itemize}

\subsection{Archiving and Preservation}
\subsubsection{Short-Term and Long-Term Strategies}
\begin{itemize}
    \item Short-term: Use networked storage or hard drives with regular backups.
    \item Long-term: Deposit data in trusted digital repositories for preservation.
\end{itemize}

\subsubsection{Key Considerations for Preservation}
\begin{itemize}
    \item Identify data that must be preserved (e.g., data underlying publications, unique datasets).
    \item Consider storage formats, metadata requirements, and legal obligations.
    \item Use repositories like Zenodo or institutional archives for long-term access.
\end{itemize}

\subsection{FAIR Principles}
\begin{itemize}
    \item \textbf{Findable:} Metadata enables search and discovery.
    \item \textbf{Accessible:} Clearly defined access conditions.
    \item \textbf{Interoperable:} Data follows standards and conventions.
    \item \textbf{Reusable:} Data is licensed and documented for future use.
\end{itemize}
\section{Data Management and Citation}

\subsection{Why Should We Cite Data?}
\subsubsection{Preventing Scientific Misconduct}
\begin{itemize}
    \item Data citation helps mitigate scientific misconduct, such as data fabrication, falsification, and selective reporting.
    \item Studies reveal that 2\% of scientists admitted to falsifying data, and 34\% acknowledged engaging in questionable research practices.
\end{itemize}

\subsubsection{Giving Credit and Recognition}
\begin{itemize}
    \item Citations are the "currency" of science, providing incentives for data sharing and acknowledging contributions.
    \item Shared datasets are cited more frequently, increasing visibility and impact.
\end{itemize}

\subsubsection{Establishing a Solid Basis}
\begin{itemize}
    \item Just as citing papers establishes a foundation for new research, data citation demonstrates a solid basis for derived findings.
    \item This practice speeds up research processes and enhances scientific discourse.
\end{itemize}

\subsubsection{Facilitating Reproducibility and Reuse}
\begin{itemize}
    \item Proper citation enables others to reproduce results, verify claims, and reuse datasets effectively.
\end{itemize}

\subsubsection{Intrinsic Benefits of Data Citation}
\begin{itemize}
    \item Data citation improves research efficiency, ensures quality, and enhances transparency.
    \item Benefits extend to easier error detection and comparability of results.
\end{itemize}

\subsection{Data Citation Principles}
\subsubsection{Key Principles}
\begin{itemize}
    \item \textbf{Importance:} Data should be treated as a legitimate, citable research product.
    \item \textbf{Credit and Attribution:} Recognize contributions through proper citations.
    \item \textbf{Evidence:} Link datasets directly to claims for validation.
    \item \textbf{Unique Identification:} Employ persistent, machine-actionable identifiers.
    \item \textbf{Access and Persistence:} Ensure long-term access to data and metadata.
    \item \textbf{Specificity and Verifiability:} Allow precise identification and validation of datasets.
    \item \textbf{Interoperability and Flexibility:} Adapt to different practices without compromising consistency.
\end{itemize}

\subsection{Data Management}
\subsubsection{What is Data Management?}
\begin{itemize}
    \item Data management involves planning and handling data throughout its lifecycle, including creation, storage, sharing, and preservation.
\end{itemize}

\subsubsection{Data Management Plans (DMPs)}
\begin{itemize}
    \item A DMP defines how data will be created, documented, accessed, shared, and preserved.
    \item Often required in academic and industrial research projects.
    \item Addresses ethics, intellectual property, and preservation strategies.
\end{itemize}

\subsubsection{Common Themes in DMPs}
\begin{enumerate}
    \item Description of data to be collected or created.
    \item Methodologies for data collection and management.
    \item Ethics and intellectual property considerations.
    \item Plans for data sharing and access.
    \item Strategies for long-term preservation.
\end{enumerate}

\subsubsection{Creating a DMP}
\begin{itemize}
    \item Researchers often complete DMP templates with questions about data handling.
    \item Machine-actionable DMPs (maDMPs) streamline planning by leveraging automation.
\end{itemize}

\subsection{Best Practices for Data Citation}
\subsubsection{Persistent Identifiers}
\begin{itemize}
    \item Use unique identifiers like DOIs or PURLs for datasets.
\end{itemize}

\subsubsection{Metadata and Standards}
\begin{itemize}
    \item Provide comprehensive metadata to describe datasets.
    \item Use community standards like Dublin Core or PREMIS to enable interoperability.
\end{itemize}

\subsubsection{Reproducibility and Traceability}
\begin{itemize}
    \item Specify dataset versions, granularity, and provenance for reproducibility.
    \item Ensure proper documentation of data subsets and usage contexts.
\end{itemize}

\subsection{Dynamic Data Citation}
\subsubsection{Challenges with Dynamic Data}
\begin{itemize}
    \item Frequent updates, corrections, and additions create challenges for precise citation.
\end{itemize}

\subsubsection{Solutions for Dynamic Data Citation}
\begin{itemize}
    \item Timestamp datasets and assign persistent identifiers (PIDs) to queries.
    \item Enable retrieval of specific data versions and subsets through queries.
    \item Provide landing pages with detailed metadata and access options.
\end{itemize}

\subsection{Archiving and Preservation}
\subsubsection{Short-Term Storage and Backups}
\begin{itemize}
    \item Use networked storage systems or hard drives with regular backups.
\end{itemize}

\subsubsection{Long-Term Preservation Strategies}
\begin{itemize}
    \item Deposit data in trusted digital repositories.
    \item Consider formats, metadata, and legal requirements for preservation.
\end{itemize}

\subsubsection{Key Considerations for Preservation}
\begin{itemize}
    \item Preserve data underlying publications, unique datasets, and historically significant information.
    \item Ensure data remains accessible for at least 10 years or per project requirements.
\end{itemize}

\subsection{FAIR Principles}
\begin{itemize}
    \item \textbf{Findable:} Metadata ensures searchability and discovery.
    \item \textbf{Accessible:} Define clear access conditions.
    \item \textbf{Interoperable:} Adhere to standards and domain-specific conventions.
    \item \textbf{Reusable:} Provide licensing and documentation to support reuse.
\end{itemize}

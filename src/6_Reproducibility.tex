\section{SHACL}
\section{Reproducibility}

\subsection{Introduction to Reproducibility}
\begin{itemize}
    \item Reproducibility is core to the scientific method and vital for ensuring the reliability and trustworthiness of research.
    \item Challenges often arise not due to misconduct but due to the complexity of processes and dynamic data environments.
    \item Ensuring reproducibility requires transparency in sharing data, code, parameters, and workflows.
\end{itemize}

\subsection{Challenges in Reproducibility}
\subsubsection{Case Studies}
\begin{itemize}
    \item \textbf{Reinhart and Rogoff (2010):} A study on debt and economic growth faced scrutiny when errors in data exclusion and statistical analysis were revealed, impacting policy debates.
    \item \textbf{Workflow Re-execution:} Analysis of scientific workflows showed a failure rate of over 75\%, highlighting issues with reproducibility in shared workflows.
    \item \textbf{Computer Systems Research:} Studies showed challenges in obtaining, building, and executing code provided by authors, with reproducibility often hindered by missing or incomplete resources.
\end{itemize}

\subsubsection{Technical and Non-Technical Issues}
\begin{itemize}
    \item Variability in implementation, software versions, and environmental conditions.
    \item Non-technical factors, such as insufficient documentation or incomplete datasets.
\end{itemize}

\subsection{Addressing Reproducibility Challenges}
\subsubsection{Standardization and Documentation}
\begin{itemize}
    \item Use standardized components, procedures, and workflows.
    \item Document the complete system setup and provenance chain to ensure traceability.
\end{itemize}

\subsubsection{Provenance Ontologies}
\begin{itemize}
    \item \textbf{PROV-O:} A W3C recommendation for representing provenance information.
    \item Includes roles, revisions, time dependencies, and plans to capture the entire research process.
\end{itemize}

\subsubsection{Automated Tools and Practices}
\begin{itemize}
    \item Automate the generation of metadata and provenance documentation.
    \item Tools like FAIR4ML and CodeMeta provide metadata schemas for machine learning and software resources.
\end{itemize}

\subsubsection{Process Capture and Preservation}
\begin{itemize}
    \item Use frameworks like the VFramework to encapsulate processes as research objects.
    \item Enable re-deployment and verification through migration and cross-compilation techniques.
\end{itemize}

\subsection{Types of Reproducibility}
\subsubsection{PRIMAD Model}
\begin{itemize}
    \item \textbf{Attributes to Prime:} Data, parameters, input data, platform, implementation, method, research objective, and actors.
    \item Gaining reproducibility in specific attributes can enhance the reliability and utility of research outcomes.
\end{itemize}

\subsubsection{Reproducibility Studies}
\begin{itemize}
    \item Focus on consistency rather than identity of results.
    \item Encourage the publication of reproducibility papers to validate prior work and foster community collaboration.
\end{itemize}

\subsection{Ethics and Privacy in Reproducibility}
\begin{itemize}
    \item Transparency in algorithmic decisions is essential to address bias and ensure accountability.
    \item Practices like data versioning and monitoring for data quality and ethical compliance are crucial.
\end{itemize}

\subsection{Learning from Non-Reproducibility}
\begin{itemize}
    \item Non-reproducibility can highlight unknown factors influencing results.
    \item Research into non-reproducible outcomes can lead to improved understanding and methodologies.
\end{itemize}
